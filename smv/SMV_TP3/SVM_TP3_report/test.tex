\documentclass{article}
\usepackage[english,francais]{babel}
%%\usepackage[utf8]{inputenc}

\usepackage[resetfonts]{cmap}

\usepackage[utf8x]{inputenc}
\usepackage[T1]{fontenc}
\usepackage[sc]{mathpazo}
\usepackage[pdftex]{graphicx}
% %\usepackage{showframe} if you want to show frame cadre de la page head ...
\usepackage{setspace}
\usepackage{hyperref}
\usepackage[french]{varioref}
\usepackage{epstopdf}
\usepackage{comment}
\usepackage{enumerate}
\usepackage{color}
\usepackage{amsmath}
\usepackage{fancyhdr}
%\pagestyle{fancy}
\usepackage{lastpage}
\usepackage{filecontents}
\usepackage{enumitem}
\usepackage{xcolor}
\usepackage{color}
\usepackage{xparse}% to define star variant of macro
\usepackage{listings}
\usepackage[sc]{mathpazo}
%\usepackage[nameinlink]{cleveref}
\usepackage{geometry}
\usepackage{filecontents}
\usepackage{xcolor}
\usepackage{xparse}
\usepackage{amsthm}
\usepackage{amsmath}
\usepackage{algorithm}
\usepackage[noend]{algpseudocode}
\usepackage{booktabs}
\usepackage{float}
\usepackage{placeins}
\usepackage{adjustbox}
\usepackage{subfig}
\usepackage{wrapfig}
\usepackage{lifecon}
\usepackage{array}
\usepackage{pdflscape}
\usepackage{adjustbox}
\usepackage{comment}
\usepackage{capt-of}
\usepackage{afterpage}
\usepackage{lscape}
\usepackage{tikz}
\usepackage{times}
\usepackage{textcomp}
\usepackage{framed,color}
\usepackage{booktabs,lipsum,calc}
%%\usepackage[font=small,skip=0pt]{caption}
\captionsetup[figure]{font=small,skip=0pt}
\captionsetup[table]{font=small,skip=-5mm}
\usepackage[final]{pdfpages}
\usepackage{booktabs}

\geometry{total={210mm,297mm},
left=25mm,right=25mm,%
bindingoffset=0mm, top=20mm,bottom=20mm}
\usepackage{hyperref}
\hypersetup{
  colorlinks   = true,    % Colours links instead of ugly boxes
  urlcolor     = blue,    % Colour for external hyperlinks
  linkcolor    = blue,    % Colour of internal links
  citecolor    = red      % Colour of citations
}
\definecolor{colour3}{rgb}{0.0,0.0,0.6} %file .adt
\definecolor{colour1}{rgb}{0.8,1.0,0.8} %to the back ground 
\definecolor{colour0}{rgb}{0.4,0.0,0.4}
\definecolor{colour2}{rgb}{0.8,0.8,0.8} %to the back ground
%\include{code}


\lstdefinestyle{MyListStyle} {
    numbers=left,
    language=Java,
    backgroundcolor={\color{yellow}},
    breaklines=true
    }

\lstdefinestyle{MyListStyle1} {
    numbers=left,
    language=Java,
    backgroundcolor={\color{colour2}},
    breaklines=true
    keywordstyle=\color{green}
    }
\lstdefinestyle{MyListStyle2} {
    numbers=left,
    language=Java,
    backgroundcolor={\color{colour0}},
    breaklines=true
    keywordstyle=\color{green}
    }    
    

\include{codegraph.adt}
\include{codeboolean.adt}

\begin{document}
% Inspiré de http://en.wikibooks.org/wiki/LaTeX/Title_Creation

\begin{titlepage}

\begin{center}

\begin{minipage}[t]{0.5\textwidth}
  \begin{flushleft}
    \includegraphics [width=5cm]{images/fac_sciences_pant.eps} \\[1cm]
    \begin{spacing}{1.5}
      \textsc{\LARGE Université de Genève}
    \end{spacing}
  \end{flushleft}
\end{minipage}
\begin{minipage}[t]{0.48\textwidth}
  \begin{flushright}
    \includegraphics [width=5cm]{images/dinfo_pant.eps} \\[1.3cm]
    \textsc{\LARGE Département d'informatique}
  \end{flushright}
\end{minipage} \\[1.5cm]

\textsc{\Large \reportsubject}\\[0.5cm]
\HRule \\[0.4cm]
{\huge \bfseries \reporttitle}\\[0.4cm]
\HRule \\[1.5cm]

\begin{minipage}[t]{0.6\textwidth}
  \begin{flushleft} \large
    \emph{Auteur:}\\
    \reportauthor
  \end{flushleft}
\end{minipage}
\begin{minipage}[t]{0.35\textwidth}
  \begin{flushright} \large
    \emph{Responsable:} \\
    M.~Jean-Henry \textsc{Morin} \\ %Enseignant:  
    %M.~Cristina \textsc{Muñoz} %Assistante: Cristina Muñoz
  \end{flushright}
\end{minipage}

\vfill

{ \today}%{\large 17 novembre 2011}

\end{center}

\end{titlepage}

\section{Introduction}

\subsection{Definition of Abstract Data Type (ADT):}

\paragraph{Abstract Data Type}is a mathematical model for data, known as \textbf{ADT}, where a data type is defined by its behavior (semantics).\\
To understand ADT, lets take two steps back, If we take off \textbf{Abstract} and Data from ADT, now we have type, and a \textbf{type} would be \textbf{defined} as a \textbf{collection} of \textbf{type values}, e.g., integer type, 0, 1, 2 etc.\\
If we add the \textbf{Data} back in we would \textbf{define data type} as  a \textbf{type} and the \textbf{set of operations} that will \textbf{manipulate the type}, e.g., addition, subtraction, and multiplication are operations that can be performed on the integer data type.\\
So, we can define an \textbf{abstract data type} (\textbf{ADT}) as a \textbf{data type}, that \textbf{is a type and the set of operations\label{operationsapp} \ref{operations} that will manipulate the type}, e.g., for the integer data type, the operations might be delete an integer, add an integer, print an integer, and check to see if a certain integer exists.\paragraph{In this TP,}we will create ADT to define oriented graphs, i.e., directed graph is an ordered pair G = (V, A) with
\begin{itemize}
\item[$\bullet$] V a set whose elements are called vertices, nodes\label{node}, or points;
\item[$\bullet$] A a set of ordered pairs of vertices, called arrows, directed edges (sometimes simply edges with the corresponding set named E instead of A), directed arcs, or directed lines.
\end{itemize} To do this, we will use an eclipse plugin called \textbf{ALPiNA}, i.e., \textbf{Al}gebraic \textbf{P}etri \textbf{N}ets \textbf{A}nalyzer and is a model checker for Algebraic Petri Nets. And with the rewriting tool in ALPiNA we will check with our solution that all the theorems defined in \textcolor{colour3}{graph.adt} are validated.
So, we will construct \textcolor{colour3}{graph.adt} file by replacing the *** with the adequate expressions.

\subsection{File structure :}
In this section i am describing each part enumerated in \textcolor{blue}{blue from 1 two 5 on left side}, and each part i explain the code line by line (you see on the left the gray background the number of the code line on the \textbf{graph.adt} file).\\
And in the TP after completing my axioms i did tests to verify it (you can find code lines and tests on the annex).
\begin{enumerate}[label=\color{blue}\theenumi]
\item \textbf{The head} of the  \textbf{graph.adt} \textit{.adt} file contains all file that we may use to construct our axioms.\\
In this TP we need to import {\textcolor{colour3}{boolean.adt}}.\label{importboolean}
\noindent
\lstinputlisting[style=MyListStyle1, linerange={2-2},firstnumber=2,]{codegraph.adt} in which we have added  the axioms to define the behavior of the operators \textbf{not} and \textbf{and} as we have seen in the exercices. I also added the axioms of the operator \textbf{or} which i want to use it after in axioms like $canReach(,,)$ and $existsCycle(,)$.
\lstinputlisting[style=MyListStyle1, linerange={19-21},firstnumber=18,]{codeboolean.adt}
\bigskip
\noindent And also in the \textbf{Head} we write the name of the \textit {.adt} file that we are workin on.
\lstinputlisting[style=MyListStyle1, linerange={4-4},firstnumber=4,]{codegraph.adt}
%\bigskip

\item \textcolor{colour0}{Sorts} \label{sort}: an ADT is a set of values, hereafter
\newpage
\item \textcolor{colour0}{Generators}\label{generator}: Define Functions and The base cases.\\
As we see on the theorems in the bottom of the file graph.adt we have 4 nodes, so we complete defining the 4 nodes A, B, C, D by replacing the *** as you see below.
\lstinputlisting[style=MyListStyle1, linerange={10-15},firstnumber=10,]{codegraph.adt}
% %<

% %>edge methode:
\bigskip\noindent
\lstinputlisting[style=MyListStyle1, linerange={18-18},firstnumber=18,]{codegraph.adt}
The method edge is to create an edge from the pairs of nodes.\\
% %>empty
\bigskip\noindent
\lstinputlisting[style=MyListStyle1, linerange={20-20},firstnumber=20,]{codegraph.adt}
The method edge is to create an edge from the pairs of nodes.\\
% %> noedge
\bigskip\noindent
\lstinputlisting[style=MyListStyle1, linerange={21-21},firstnumber=21,]{codegraph.adt}
\label{noedge}I want to use this on the axiom getEdge \ref{axiomgetedge}\\

% %> cons
\bigskip\noindent
\lstinputlisting[style=MyListStyle1, linerange={22-22},firstnumber=22,]{codegraph.adt}
We want to define inductivly a graph so the result of cons should be graph.\\
and as we on the L 49 \ref{cons} cons take $\$e1$ so edge as first parameter and graph $\$g$ as second parameter.



\bigskip\noindent
\item \textcolor{colour0}{Operations\label{operations} \ref{operationsapp}}: As we talked above, the set of operations is to manipulate the type, for the edge data type, the operations might be delete an edge, add an edge etc...

\lstinputlisting[style=MyListStyle1, linerange={24-24},firstnumber=24,]{codegraph.adt}
\ref{operations}.
\lstinputlisting[style=MyListStyle1, linerange={26-26},firstnumber=26,]{codegraph.adt}
The source of the edge is a node, from where the edge is starting, throughout his direction.
\bigskip\noindent
\lstinputlisting[style=MyListStyle1, linerange={28-28},firstnumber=28,]{codegraph.adt}\label{targetedge}
The target of an edge is a node so this functions return the node, to where the edge is pointing and it take an edge as parameter.
\bigskip\noindent
\lstinputlisting[style=MyListStyle1, linerange={30-30},firstnumber=30,]{codegraph.adt}
This function remove an edge from the graph so it take as parameter a graph and an edge.
\bigskip\noindent
\lstinputlisting[style=MyListStyle1, linerange={32-32},firstnumber=32,]{codegraph.adt}
This function as we see at the first sight in \textbf{Line 53} return false so it return bool and also the second parameter is node to it take $\$x$ as second parameter.
and that mean that if the graph is empty, then no terminal for the graph.

\bigskip\noindent
\lstinputlisting[style=MyListStyle1, linerange={34-34},firstnumber=34,]{codegraph.adt}
This function test if there are an edge (true) from a node or not (false) so it take graph and node as parameter and it return boolean.

\bigskip\noindent
\lstinputlisting[style=MyListStyle1, linerange={36-36},firstnumber=36,]{codegraph.adt}
Given a node it return the first edge so the second parameter is node.

\bigskip\noindent
\lstinputlisting[style=MyListStyle1, linerange={38-38},firstnumber=38,]{codegraph.adt}
Given the theorem below of line 77 \vspace{-2mm}
\lstinputlisting[style=MyListStyle, linerange={93-93},firstnumber={93}]{codegraph.adt}
we see that it return boolean because it will test if a node can be reached from itself \label{canReachItself} \ref{axiomcanReach} or from another given node(true) or not. and it takes as parameter node, node, graph (we define it respecting the given order)
\bigskip\noindent

\lstinputlisting[style=MyListStyle1, linerange={40-40},firstnumber=40,]{codegraph.adt}
Given the theorem below of line 96
\lstinputlisting[style=MyListStyle, linerange={96-96},firstnumber={96}]{codegraph.adt} we see that it return boolean and it takes as parameter node, node, graph (we define parameters respecting the given order), so it Test if a node belongs to a cycle (true) or not(false).\\



\bigskip\noindent



\item \textcolor{colour0}{Axioms}\label{axiom}: Defines relations between operations, a self-evident truth, or a proposition whose truth is so evident as first sight.


\bigskip\noindent
\lstinputlisting[style=MyListStyle1, linerange={44-44},firstnumber=44,]{codegraph.adt}
the source of an edge is $x$, so it return $\$x$.
\bigskip\noindent
\lstinputlisting[style=MyListStyle1, linerange={45-45},firstnumber=45,]{codegraph.adt}
The target of an edge is the second parameter, so It return the node in which is the direction of this edge \ref{targetedge}, so it return $\$y$.
\bigskip\noindent
\lstinputlisting[style=MyListStyle1, linerange={49-49},firstnumber=49,]{codegraph.adt} \label{cons}
if $e = e1$ It remove the edge e and it return a graph.
as we see in the line below
\vspace{-2mm}
\lstinputlisting[style=MyListStyle, linerange={48-48},firstnumber=48,]{codegraph.adt}\vspace{-3mm}so there is no change for the graph even if we delete an edge, so it return $\$g$

\bigskip\noindent
\lstinputlisting[style=MyListStyle1, linerange={50-50},firstnumber=50,]{codegraph.adt}
if $e$ different then $e1$ so adding $e1$ to $g$ then remove e, is the same as remove $e$ from $g$ and then add $e1$ .
\bigskip\noindent

\lstinputlisting[style=MyListStyle, linerange={53-53},firstnumber=53,]{codegraph.adt}\lstinputlisting[style=MyListStyle1, linerange={54-55},firstnumber=54,]{codegraph.adt}
The given case, line 53, we see that if no graph then retrun false ( node is not terminal). So here there is an essential condition should be considered, i.e., the graph is not empty otherwise the test $isTerminal(\$g, \$x)$ will give us false all times.
And by drawing some graph, we conclude that a node can be terminal  $if\ and \ only \ if $ it's not a source for any edge, that's why i added the second condition (existsEdge(\$g,\$x) =true) and for the $2^{nd} $ case, if node is a source for one of edges then the node is not terminal. Also,
we can achieve this axioms in another way using $getEdge(\$g,\$x)$.
\bigskip\noindent
\lstinputlisting[style=MyListStyle1, linerange={59-60},firstnumber=59,]{codegraph.adt}
L59, It retrun $noedge$ \ref{noedge} if the edge do not exist.\\
L60, If $\$x$ is the source of $e$ then it will return the edge which has this node $x$ as a source \label{axiomgetedge}

\lstinputlisting[style=MyListStyle1, linerange={61-61},firstnumber=61,]{codegraph.adt}
If $\$x$ is not the source of $e$ then it will go recursively throughout the graph but \{ \textit{\textcolor{red} {if we can say}} the $new\ graph$ or $reduced\ graph$ \} do not contains the edge $e$, which $e$ has as source x, so it's like going throughout the rest of the reduced graph from the edge $e$.

\bigskip\noindent
\lstinputlisting[style=MyListStyle1, linerange={67-67},firstnumber=67,]{codegraph.adt}
If the no graph then no existing edge then return false.
\lstinputlisting[style=MyListStyle1, linerange={68-68},firstnumber=68,]{codegraph.adt}
If $\$x$ is the source of $e$ then the existsEdge function will return true because the graph contain the edge $e$ which has this node $x$ as a source.

\bigskip\noindent
\lstinputlisting[style=MyListStyle1, linerange={69-69},firstnumber=69,]{codegraph.adt}
If $\$x$ is not the source of $e$ then the existsEdge function will go recursively throughout the graph $g$ until it find the edge otherwise it will return false if there are no edge corresponding to this edge $e$.
The mechanism is similar to getEdge. 

\bigskip\noindent
\lstinputlisting[style=MyListStyle1, linerange={75-75},firstnumber=75,]{codegraph.adt}
As we have seen above, a node can reach itself \ref{canReachItself}\label{axiomcanReach} so considering the condition (same node) it return true.

\lstinputlisting[style=MyListStyle1, linerange={76-76},firstnumber=76,]{codegraph.adt}
if there are no edge then $x$ cannot reach $y$, and in this case there is no graph $ canReach(\$x,\$y,\textcolor{red}{empty})$ then no edges then the result will be false.

\bigskip\noindent
\lstinputlisting[style=MyListStyle1, linerange={77-77},firstnumber=77,]{codegraph.adt}
if $\$x$ is different then $\$y$ and the graph is not empty and there are no existing edge for $\$x$, \textit{in other words} x is not the source for no one edge, then it cannot reach the node from $x$ and the result of test is false.
and in another way to do it, if $x$ is terminal also cannot be reached from another node (not itself).
\bigskip\noindent
\lstinputlisting[style=MyListStyle1, linerange={78-78},firstnumber=78,]{codegraph.adt}
if $\$x$ diffrent then $\$y$ and $\$g$ not empty and $\$x$ is a source of an edge then we do the test recursivly because we will be in front of many cases.\\
At the first time if we get the first edge $canReach(target($getEdge(\$g, \$x)$), .., ..)$ and we find that we cannot reach this node $\$x$ from $\$y$ then the result will be false! but no, here appear the second case that with from the node $\$x$ which is source of the edge $e$(first edge) we cannot reach it and maybe there are another way to reach it given another node $\$y$ .\\
so what i am doing next is to remove the edge the i passed ( just before), reducing my graph (obtaining graph without the passed edge) \\
and do the test again and again so the test will will pass by the axioms of canReach L-74, 75, 76 and it will return the corresponding result. true if the first case was founded and false in the rest.

\bigskip\noindent
\lstinputlisting[style=MyListStyle1, linerange={85-85},firstnumber=85,]{codegraph.adt}
If there is no edge given a node $\$x$ result is false, \textit{(no edge mean graph empty  so no cycle because no edge)}.
\bigskip\noindent
\lstinputlisting[style=MyListStyle1, linerange={86-86},firstnumber=86,]{codegraph.adt}
In this axioms we tread all cases.\\
If there is an edge starting from the node $\$x$ then i am testing if i's reachable from the target of the edge of $\$x$, $\$x$ is source\\
And in $canReach(,$\$x$,)$ the second parameter is $x$ so it means that $x$ is also the destination of the edge then there are a cycle. \\ 
But in case if $\$x$ is not target of the first edge, so here the third case appear and the test go recursively (same to can reach) , $->$ removing the passed edge, $->$reducing my graph, and $->$ do the test of existsCyle again until it find the source is the same as the destination in the new graphs, otherwise the result will be false.






\item \textcolor{colour0}{Theorems}\label{theorems}: A theorem is a statement that can be demonstrated to be true by accepted mathematical operations and arguments.

\item \textcolor{colour0}{Variables}\label{variables}: It's a generic element of sorts.
\end{enumerate}
\subsection{Graph ADT Construction}
\noindent
% % Showing line range x,y ,z..
\lstinputlisting[style=MyListStyle1, linerange={9-22},firstnumber=0,]{graph.adt}
    
\bigskip\noindent
\lstinputlisting[style=MyListStyle1, linerange={8-8},firstnumber=9,]{graph.adt}

\bigskip\noindent
\lstinputlisting[style=MyListStyle1, linerange={8-8},firstnumber=9,]{graph.adt}




%%Showing lines 1,3,7,12 (note that Line 7 is blank) with starred version between lines 1 and 3 to supress the space and the space before line 12 set to 50\% of the \verb|\baselineskip|:

%%\NewDocumentCommand{\ShowListingForLineNumber}{s O{1.0} m m}{
%%    \IfBooleanTF{#1}{\vspace{-#2\baselineskip}}{}
%%    \lstinputlisting[
%%            style=MyListStyle1,
%%            linerange={#3-#3},
%%            firstnumber=#3,
%%            ]{#4}
%%}%
%%
%%\ShowListingForLineNumber{3}{graph.adt}% supress space before
%%\ShowListingForLineNumber{7}{graph.adt}% supress 50% of the space before
%%\ShowListingForLineNumber{12}{graph.adt}
%%\lstinputlisting[style=MyListStyle1, linerange={42-42},firstnumber=42]{graph.adt}

\end{document}