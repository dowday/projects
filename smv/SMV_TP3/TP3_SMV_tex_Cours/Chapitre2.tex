\chapter{La M\'ethode Tabou}

\section{intro}
(Tabu secret)

Propos\'e par glover en 1986 
Une m\'ethode bien adopter au probl\'eme d'\underline{affectation quandution}(QAP)
L'espace de recherche est donc l'espace des permutation de  $n$ objets.
Le contexte tabu ob\'eit aux  principe de base on explore it\'erativement $S$
\begin{center}
$X1 --U--> X1 \in V(x_0) ---U-->...$
\end{center}
o\`u U est sp\'ecifique \'a la recherche  tabu

U va choisir dans $V(x_{n})$le meilleur Xmax
o\`u meilleur signifi:
\begin{itemize}
\item Xmax n'est pas \underline{tabus} donc que Xmax est une solution permise.
\item de fitness meilleur dans $V(x_b)+{X_n}$ donc le meilleur voisin n'est pas interdit.
\item s'il ya plusieurs voisin de m\^eme fitness optimale dans $V(_{x_n})$ on en choisit un au hasard.
\end{itemize}
La m\'ethode n\'ecessite d'explor\'e tout le voisinage

La s\'ecificit\'e de la m\'ethode tabu est l'existence d'une liste de permets interdits (tabus) dont le but est d'emp\^eches la recherche de repasser par des solutions d\'ej\'a vue.

Le contenue de la liste tabou n'est pas permanent, mais en constante \'evolution: on y ajoute et enl\`ve constament de l'information.


$(\textcolor{red}{check- diagram - 11})$ 


$(\textcolor{red}{check- example 1/4 , in course pdf - 11})$ 

\section{Convergence de la m\'ethode}
Par convergence, on veut donc la propri\'et\'e de trouver l'optimum global
La r\'eponse est oui si:
\begin{itemize}
\item L'espace $S$ est fini, le voisinage est sym\'etrique (si $x in V(y)$ alors $y \in V(x)$ et que tout point de $S$ s'atteind de tout autre  par un nb fini de mouvements.
\item La liste tabou m\'emorise tous les points visit\'es tout en permettant de repassser par le plus ancien point tabou s'il n'y a plus d'option

Alors l'espace est visibl\'e de facon exhoustive et qu'on trouvera l'optimum.
Mais ce r\'esultat n'est {gu\'`re} utile car il pr\'edit un \underline{temps exponentiel} dans la taille de $S$ pour tout visiter.

\subsection{La liste Tabou}
Que mettre dans la liste tabout ?
il y a plusieurs strat\'egies:
\begin{itemize}
\item les point d\'ej\`a visit\'es 
\item Des attributs des points visit\'es 
\item l'inverse des mouvement utilis\'es.

\end{itemize}
Pour garder les point visit\'es, il faut sp\'ecifier la taille m\'emoire M et avoir un buffer circulaires: 

$(\textcolor{red}{check- dessin - 12})$ 

On peut aussi garder des attribut des solution 
Soit $T$ un tableua de taille $M$ et $h(x)$ un propri\'et'e (un attribut) de x que l'on supoose entier.
On peut calculer $h(x)$ mod $M$ et dire que cette valeur est interdite \underline{pendant t it\'erations } t s'appelle la \underline{dur\'ee de l'interdiction}
Si un tel $x$ est visit\'e \'a l'interdiction k on placera en 
\begin{center}
$T[h(x) \mod M]= K+t$

\end{center}
Un point $X'$ candidat comme successuer pour l'exploration sera i \'a l'it\'eration k' si 
\begin{center}
$K' < T[h(x) \mod M]$
\end{center}
Le principe de la dur\'ee de l'interdicction est beaucoup utilis\'e notament si la liste tabou contient l'invers de mouvement effectu\'e c'est la solution la plus courante \\
$check tab 13$\\
C'est le movement inverse qui est tabou on veut eviter de reunir la ou on visiter 

$check tab 14$
La m\'emoire \'a ling terme fait des statistiques sur les fr\'equence des mouvement choisis et peut prendre le contr\^ole de la m\'emoire \'a cout terme (dur\'ee d'interdiction) et forcer  des mouvement bcp  trop plus employ\'es (utili\'es)

\end{itemize}

\subsection{Liste Tabou}
Liste tabou: point, mouvement ou attribut interdit.\\ Un ingrédient important est la durée de l'interdiction.\\ \\
On se souvient que les métaheuristiques utilisent deux principes:\\
- la diversification (exploration)\\
- et l'intensification (exploitation). \\

Si la liste tabou est courte, on choisit aussi les durées d'interdictions courtes, on favorise l'intensification car les points de bonnes fitness dans le voisinage sont rarement tabous. \\
Dans le cas contraire, on va explorer des régions de bonnes fitness et favoriser la diversification.\\

\paragraph{\underline{Notation d'aspiration :}} La méthode tabou vient aussi avec un mécanisme dit d'aspiration:\\
Une solution tabou peut être choisie si elle est meilleure que toutes les solutions rencontrées jusqu'à présent (si on passe proche de qqch de très bien, on vas y aller).
\textcolor{red}{La m\'ethode tabou vient aussi avec un m\'ecanisme dit d'\underline{aspiration}
Un solution tabou peut \^etre choisie si elle est meilleure que toutes les solutions rencontr\'ee jusqu'\'a pr\'esent.} -> this is my copying from professor 
\section{Probl\`eme d'affectation quadratique}
C'est une famille de probl\'eme d'optimisation combinatoire tr\'es fr\'equente dans la pratique \\
\begin{itemize}
\item On se donne n objets et n emplacements pour ces objets.
\item On se donne des flux $f_{ij}$ entre chaque paire i et j d'objets.
\item On se donne une distance $d_{rs}$ entre chaque emplacement r et s
\end{itemize}

\boldmath{schema} \textcolor{red}{voir photo pris de anh}
Le voyageur de commerce est aussi un probl\`me QAP

l'espace de recherche S, l'ensemble des permutations de n objets \\
$p=(i_1,i_2,i_3...i_n))$ $i_k$ num\'ero de l'objet positionner en k\\
ou espace $p'=(p_1,p_2,..,p_n)$ avec $p_{r_k}$ la posistion de l'objet $k$\\
La fitness du probl\'eme, auon veut minimiser est :
\begin{center}
%$F=\sum_(ij)F_i_j d_{r_i}_{r_i} = \sum_{r,s}$  
$F=\sum_{ij} F_{ij} d_{{r_i}_{r_i}} $
somme sur toutes les paires d'objet \\
$d_{{{r}_{i}}{{r}_{j}}}$

\textcolor{red}{copier ce que j'ai rater dans le debut de la seance voir slack}
\end{center}

La liste tabou est une matrice $T$ de taille $nxn$ tel que l'\'el\'ement $T_ir$ est le dernier mouvement(it\'eration) o\`u l\'objet i a quitt\'e l'emplacement r, additionn\'e d'une dur\'ee d'interdiction t al\'eatoire.
A chaque it\'eration il y a 2objets qui bougent ce qui affecte deux \'el\'ements de T

Le mouvement $(r,s)$ sera tabou \`a l'it\'eration k si \'a la fois  $NOTE Corrige les faute commente $

% $T_{i_{s}_r}$ et  $T_{i_{r}_s}$ 
contenant une valeur sup\'erieur \'a $k$
En d'autres termes on n'a pas le droit de replacer $x$ objets donn\'es \'a des emplacement qu'ils ont d\'ej\'a occup\'e dans le pass\'e  (\'a une distance t de l'it\'eration courante)
\newpage
