\chapter{Introduction sur la M\'etaheurisitque}
\centerline{MHOA, Monday September 28} % Pas de numérotation
%\addcontentsline{toc}{section}{Introduction} % Ajout dans la table des matières
\begin{itemize}
\section{\underline{$S$ Espace de recherche} }

\item  Comme la heuristiques , le metaheuristiques n'offrent pas de garantie de suces.
donc pour les problems pour lequels il existe un algorithmes exactes, il faut les utiliser et non les m\'etaheuritiques

\section{\underline{Caract\'eristiques de m\'etaheuristiques}}

\item On ne fait aucune hypothèse sur les propriétés mathématique de la fitness(continuité, dérivabilité)
il faut seulement pouvoir calculer f en but $ x \in S$

\item Elles utilisent des param\`etres de guidage on ne connait pas les valeurs optimiste de ces param\'etres qui augmente la qualit\'e de la solution et diminue le temps de calcul.

\item Elles utilises des nombres al\`eatoires lors de l'expploration de $s$.
Ce sont donc des m\'ethodes statistiques.

\item Elles sont gourmande en CPU, mais facile \'a impl\'ementer, et plusieurs se parall\'elisent facilement.

\item Elles n\'ecessitent une condition initiales pour explorer $s$ souvent elle est choisie au hasard.
\item N\'ecessitent une condition d'arr\^et.
Souvent c'est une limite du temps de calcule \boldmath{OU} encore une stagnation de la $Fitness$ de la meilleur solution trouv\'ee plus de progr\`es enregistr\'e
\item Les m\'etaheuristiques combinent deux m\'ecanismes dans leur exploration :
\begin{enumerate}

\item \underline{intensification (ou exploitation)} on essaye d'am\'eliorer une solution pormeteuses]
\item \underline{diversification (ou exploration)} on visite de nouvelles r\'egimes de de l'espace de trouver de meilleurs solutions
\end{enumerate}
\section{Principe de fonctionnement}

On se d\'eplace de proche en proche de $S$ (Check dessin 1)

On doit donc d\'efinir un voisinage qui sp\'ecifie les points candidat pour la suite de l'exploration.

$(\textcolor{red}{Check the formula 2 and write it})$ avec $U$ l'op\'erateur de recherche

\section{\underline{Comment peut on sp\'ecifi\'e le voisinage}}

Naivement, on pouvait penser \'a une liste de voisins pour chaques point de $S$
Pour des S qund, ce n'est pas possible.
on va p\'ef\`erer une \'ecette qui perment de g\'en\'erer le voisinage de n'importe quel $x \in S$
En pratique on d\'efinie les transformations (ou mouvement )
$(\textcolor{red}{Check the formula 3 and write it})$

$(\textcolor{red}{Check the formula 4 and write it})$

\textcolor{green}{Certaines m\'etaheuritstiques  g\'en\`erent un voisin au hasard, en g\'en\`erale en tenant compte de la quantit\'e $\underbrace{(au  sens de Fitness)}$ de ce voisin }

On peut Imaginer des voisinages heuristiques
$(\textcolor{red}{Check the formula 5 and write it})$

si aucun voisin dans $Vi(x)$ n'est int\'eressant (meilleur aue ce qu'on a d\'eja), on explore $Vi=1$. Si, \'a la fin de la hierarchie, on n'a rien trouv\'e, on s'arr\^ete.
\subsection{Exemples d'eploration simples}
\item  
\paragraph*{Recherche al\'eatoire :} $U$ choisit au hasard dans $V(x)$ le successeurm ind\'ependament de la valeur de la $FITNESS$ \\
- Si $V(x) = S$ on a la recherche al\'eatoire proposant  dite.
\item Si $V(x)$ est local  \'a chaque $x$ on a ce qu'on appele une marche al\'eatoire 
$(\textcolor{red}{Check the formula 6 and write it})$
\item\underline{Grimpeur Strict(Hill climbing)}
$U$ le voisin de fitness optimale dans $V(x)$. Avenc le risque au'on se bloque si on est d\'ej\'a \'a l'optimum.
En g\'eneral, cela m\`ene \'a un optimum local

\item \underline{Grimpeur probabiliste}
On choisit un voisin au hasard avec une probabilit\'e d'autant plus grande que la fitness est bonne.
Cela perment a priori de s'echapper d'un optimum

\subsection{Exemple le maxOne}
Il faut maximise rle nomber de 1 une chaine binaire.
On va voir que le succ\'es d'une m\'etaheuritique d\'epend grandement du choix du codage du probl\'eme et des choix du voisinage.
On va illustrer ce fait avec le grimpeur strict.
Soit une cha\^ine de 3 bits $X1X2X3$
On peut repr\'eenter l'espace de recherche comme l'ensemble des nombre entre 0 et 7 
000 --> 0
001 --> 1
. . . 
111 -->7
$S = {0, 1, 2, ..., 7}$
On d\'efinit dans le voisinage par exemple $V(x)={x-1, x, x+1}$\\
$(\textcolor{red}{Check the formula 7 and write it})$\\
\begin{lstlisting}
\textcolor{red}{
|
|
|                            .(max globale)
|           .       .   .   
|   .   .       .   
--------------------------------
    1   2   3   4   5   6   7
    text}
\end{lstlisting}
La fitness est le nb de bit \'a 1 dans x

\item\underline{Autre repr\'esentation: } L'espace de recherhce est ${0,1}^3$
donc on garde la repr\'esentation binaire.
la topologie du voisinage est contruite en utilisant une repr\'esentation en hypercube.\\
$(\textcolor{red}{Check the formula 8 and write it})$

Ce voisinage a la propri\'et\'e que les solutions sont voisinges si elles ne diff\'erent pas que par 1 bit.

\section{Paysage de Fitness}
(ou fitness landscape ou essayy landscape)
C'est la repr\'esntation de la fonction fitness en fonction des point de S, et representatnt la topologie par le voisinage.\\
$(\textcolor{red}{Check the formula 9 and write it})$

On ne eut en g\'en\`ral pas dominer de facon {connait} le paysage de fitness.
Mais ce qui le caract\'erise c'est l'importance des  {vancatins} de F entre 2 ponts voisins de S.

\section{M\'ethode \'a population}
Ce sont de m\'etaheuristiques qui consid\'ere \'a chque it\'eration un ensemble de solution possible (une population de slolution), \'a l'oppos\'e de m\'etaheuritique \'a 1 individu comme de suite pr\'ec\'edemment\\
$(\textcolor{red}{Check the formula 10 and write it})$\\
avec une population, on n'a pas de chance d etrouve une bonne solution mais cela perment plus de temps CPU

la g\'en\'eraton \'a partir de PI se fait par une transformation qui d\'epend de tous les \'el\'ement de $Pi$. Voir par exemple les \underline{Algorithmes g\'en\'etiques}
Formellemtn, l'espace de recherche devient \\ 
\begin{center}
$S^n=SxSxSxS...xS$
\end{center}
avec n individus qui \'evaluent en m\^eme temps.
\end{itemize}