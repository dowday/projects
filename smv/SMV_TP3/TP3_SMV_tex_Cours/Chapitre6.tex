\chapter{Algorithmes evolutinnaires}

\section{Introduction}

Les algo \'evolutionnaires sont inspir\'es de l\'evolution darwinienne des esp\`eces vivante.


Par croisement et mutation du patrimoine g\'en\'etique, l\'evolution cr\'ee des populations adapt\'ees \'a peu environnement.

Ici l'environnement sera d\'ecrit par la fonction fitness et le patrimoine g\'en\`etique par le codage des solution.
\textcolor{green}{
Cela donnes des m\'etaheuriques \underline{\'a population}  qui ont apparu dans les ann\'ees 70.\\
\begin{itemize}
\item Algorithmes g\'en\`etique (J. Hollnad)
\item Start\'egie d\'evolution (I. Rechnberg)
\item Programmation g\'en\`tique (L. J. Fogel, JR KOZA)
\end{itemize}
				}
\section{Algo Ge\'en\'etique}
On se donne une population d'individus qui correpondent \'a une solution au prob\'eme.
Donc chaque individu est un point de l'espace de recherche S.

La repr\'esentation d'un individu dans S(\eg  un vecteur de bool\'een) \\
est ce qu'on appelle ici son patrimoine g\'en\'etique ou encore son chromosome.

Le degr\'e d'adaptation de l'individu sera donn\'e par sa fitness. \\L\'evolution tend \'a produire des individu de haute fitness, donc on pense ici \'a des probl\'emes de maximisation


D\'eroulement de l'\'evolution par un algo gen.

\begin{itemize}
\item A chaque iteration(souvent appelée génétiaues) on vq effectuer les opérqtions suivqntes:
\begin{itemize}
\item Selection des meilleurs individus pour une phase de reproduction.
\item Croisements de paires d'individus selectionn\'es $->$ pour produire la g\'en\'eration suivante.
\item Mutation des enfants pour produire la g\'en\'eration suivante.
\end{itemize}
\end{itemize}
On impose que la taille de la population reste constante au cours de l'\'evolution ceci pour des raisons pratique. Cela impose des restrictions sur les op\'erateurs de selection et croisement

Souvent, pour ne pas perdre la meilleur solution trouv\'ee, le meilleur individu est automatiquement ins\'er\'es dans la g\'en\'eration suivante.




\begin{algorithm}
\caption{My algorithm}\label{euclid}
\begin{algorithmic}[1]
%\Procedure{}{}
\State $\textit{generation = 0}$
\State While (not end condition)
\State 	\hspace{2cm}{$generation\ +=1$}
\State  \hspace{2cm}{compute fitness of each individual}
\State  \hspace{2cm}{select individual}
\State  \hspace{2cm}{crossover}
\State  \hspace{2cm}{mutation}
\State End While
%\EndProcedure
\end{algorithmic}
\end{algorithm}

Population $P(t)$ \\
$P(t)  \overrightarrow{select} P'(t) \overrightarrow{crossover} P''(t)\overrightarrow{mutation} = P(t+1)$

Les populations sont en general de taille n avec n entier quelques dizaines ou quelque certaines\\
la population initale est tiree au hasard .. (reste copier de X)\\

\underline{Operateur de Selection}\\
Comment passse-t-on de $P(t)$ \`a $P'(t)$ ?\\

\underline{On tire avec remise } $n$ fois
Un individu parmi les $n$.\\ on a donc le droit de tirer plusieurs fois le m\^eme individu.\\Ce tirage se fait proportionnelemtn \`a la fitness. On selectionnera plus volontier les individus de bonne fitness.\\
Ces $n$ individus ainsi selectionn\'es seront les parent de la prochain g\'en\'eration.\\
\underline{(Operateur de crossover)}\\
Comment passer de $P'(t)  -> P''(t)$\\
avec   $P'(t)$ : parent et   $P''(t)$: enfants.\\
On forme $\frac{n}{2}$ paire de parents.\\
\eg  on associes les individus $S_i, i=12... n $ ainsi:\\
\textcolor{green}{premier couple} $(S_1 \ S_2) (S_3 \ S_4) ... (S_{n-1} \ S_n)$ \textcolor{green}{dernier couple de parent}\\
Avec probabilit\'e $P_{crossover}$ chaque couple donne 1 enfant et avec $P = 1-P_crossover$, les 2 parent restent\\
\begin{center}
\[ (S_i \ S_{i+1})-> \left\{
                \begin{array}{ll}
                  S'_i,S'_{i+1}  \ avec\ prob\ P\ crossover \\
                  S_i, S_{i+1} 	 \ avec \ 1- \ Pcrossover\\
                  
                \end{array}
              \right.
  \]
\end{center}

On verra, selon le codage des individus, comment on construit les enfants \`a partir des parent. Ce srea en "croisant" les codages.\\
Cette nouvelle population $P''(t)$ est alors pr\^ete pour la derni\'ere phase, les mutation.\\

\underline{Operateur de mutation}\\
$P''(t) -> P'''(t) = P(t+1)$\\
chque individu de $P''(t)$ est soumis \`a mutation.\\
Avec proba $P_{mutation}$ chque composante de chaque individu est modifi\'ee al\'eatoirement.

Apr\'es cette \'etape, on obtient la nouvelle population $P(t+1)$ \\
La selection favorise les bons individus et donc favorise l'intensification.\\
Le crossover et la mutation permettent de c\'eer de nouvelles solutions.\\
Ils favorisent la dversification. cette derni\`ere est amplif\'ee avec des grandes prob $P_{crossover}$ et $P_{mutation}$.


\hfill\LARGE\textbf{end session of 26 oct 2015}