
\definecolor{codegreen}{rgb}{0,0.6,0}
\definecolor{codegray}{rgb}{0.5,0.5,0.5}
\definecolor{codepurple}{rgb}{0.58,0,0.82}
\definecolor{backcolour}{rgb}{0.95,0.95,0.92}
\definecolor{dkgreen}{rgb}{0,0.6,0}
\definecolor{gray}{rgb}{0.5,0.5,0.5}
\definecolor{mauve}{rgb}{0.58,0,0.82}

\lstset{frame=tb,
  language=Java,
  aboveskip=3mm,
  belowskip=3mm,
  showstringspaces=false,
  columns=flexible,
  basicstyle={\small\ttfamily},
  numbers=none,
  numberstyle=\tiny\color{gray},
  keywordstyle=\color{blue},
  commentstyle=\color{dkgreen},
  stringstyle=\color{mauve},
  breaklines=true,
  breakatwhitespace=true,
  tabsize=3
} 
\lstdefinestyle{mystyle}{
    backgroundcolor=\color{yellow},   
    commentstyle=\color{codegreen},
    keywordstyle=\color{magenta},
    numberstyle=\tiny\color{codegray},
    stringstyle=\color{codepurple},
    basicstyle=\footnotesize,
    breakatwhitespace=false,         
    breaklines=true,                 
    captionpos=b,                    
    keepspaces=true,                 
    numbers=left,                    
    numbersep=5pt,                  
    showspaces=false,                
    showstringspaces=false,
    showtabs=false,                  
    tabsize=2
}

\lstset{style=mystyle}
\lstdefinestyle{MyListStyle} {
    numbers=left,
    language=Java,
    backgroundcolor={\color{yellow}},
    breaklines=true
    }
\geometry{total={210mm,297mm},
left=25mm,right=25mm,%
bindingoffset=0mm, top=20mm,bottom=20mm}
\usepackage{hyperref}
\hypersetup{
  colorlinks   = true,    % Colours links instead of ugly boxes
  urlcolor     = blue,    % Colour for external hyperlinks
  linkcolor    = blue,    % Colour of internal links
  citecolor    = red      % Colour of citations
}
\definecolor{colour3}{rgb}{0.0,0.0,0.6} %file .adt
\definecolor{colour1}{rgb}{0.8,1.0,0.8} %to the back ground 
\definecolor{colour0}{rgb}{0.4,0.0,0.4}
\definecolor{colour2}{rgb}{0.8,0.8,0.8} %to the back ground
\definecolor{DarkGreen}{rgb}{0.0,0.4,0.0} % Comment color
\definecolor{highlight}{RGB}{255,251,204} % Code highlight color
\definecolor{blue}{rgb}{0.0,0.2,1.0}
\lstdefinestyle{MyListStyle} {
    numbers=left,
    language=Java,
    backgroundcolor={\color{white}},
    breaklines=true
    }

\lstdefinestyle{MyListStyle1} {
    numbers=left,
    language=Java,
    backgroundcolor={\color{colour2}},
    breaklines=true
	keywordstyle=\color{green}
	}


\lstdefinestyle{MyListStyle2} {
    numbers=left,
    language=Java,
    backgroundcolor={\color{colour0}},
    breaklines=true
    keywordstyle=\color{green}
    }

\lstdefinestyle{Style1}{ % Define a style for your code snippet, multiple definitions can be made if, for example, you wish to insert multiple code snippets using different programming languages into one document
language=matlab, % Detects keywords, comments, strings, functions, etc for the language specified
backgroundcolor=\color{backcolour}, % Set the background color for the snippet - useful for highlighting
basicstyle=\footnotesize\ttfamily, % The default font size and style of the code
breakatwhitespace=false, % If true, only allows line breaks at white space
breaklines=true, % Automatic line breaking (prevents code from protruding outside the box)
captionpos=b, % Sets the caption position: b for bottom; t for top
commentstyle=\usefont{T1}{pcr}{m}{sl}\color{dkgreen}, % Style of comments within the code - dark green courier font
deletekeywords={}, % If you want to delete any keywords from the current language separate them by commas
%escapeinside={\%}, % This allows you to escape to LaTeX using the character in the bracket
firstnumber=1, % Line numbers begin at line 1
frame=single, % Frame around the code box, value can be: none, leftline, topline, bottomline, lines, single, shadowbox
frameround=tttt, % Rounds the corners of the frame for the top left, top right, bottom left and bottom right positions
keywordstyle=\color{blue}\bf, % Functions are bold and blue
morekeywords={}, % Add any functions no included by default here separated by commas
numbers=left, % Location of line numbers, can take the values of: none, left, right
numbersep=10pt, % Distance of line numbers from the code box
numberstyle=\tiny\color{codegray}, % Style used for line numbers
rulecolor=\color{black}, % Frame border color
showstringspaces=false, % Don't put marks in string spaces
showtabs=false, % Display tabs in the code as lines
stepnumber=5, % The step distance between line numbers, i.e. how often will lines be numbered
stringstyle=\color{codepurple}, % Strings are purple
tabsize=2, % Number of spaces per tab in the code
}

% Create a command to cleanly insert a snippet with the style above anywhere in the document
\newcommand{\insertcode}[2]{\begin{itemize}\item[]\lstinputlisting[caption=#2,label=#1,style=Style1]{#1}\end{itemize}} % The first argument is the script location/filename and the second is a caption for the listing

%----------------------------------------------------------------------------------------
   
\newtheorem*{remark}{Remark}
\fancyhf{}
\rhead{Analyse et Traitement de l’Information}
\lhead{Oday Darwich}
%\rfoot{Pags\ \thepage \ of \ \pageref{LastPage}}
\rfoot{Page\ \thepage} 
\newcommand{\eg}[0]{\mbox{e.g. }\xspace}

\makeatletter
\def\BState{\State\hskip-\ALG@thistlm}
\makeatother

% % Exercice question answer envirement 
\renewcommand{\QuestionNB}{Question~\arabic{Question}.\ }
\setlength{\QuestionIndent}{7em}
